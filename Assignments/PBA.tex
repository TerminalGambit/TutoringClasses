\documentclass[a4paper,12pt]{article}
\usepackage[utf8]{inputenc}
\usepackage[T1]{fontenc}
\usepackage[french,english]{babel}
\usepackage{amsmath,amsfonts,amssymb}
\usepackage{tcolorbox}
\usepackage{listings}
\usepackage{graphicx}
\usepackage{lmodern}
\usepackage{enumitem}
\usepackage{geometry}


\title{Python Learning Path: Project-Based Approach}
\author{Prepared by Jack Massey}
\date{}

\lstset{
  basicstyle=\ttfamily\small,
  backgroundcolor=\color{gray!10},
  frame=single,
  columns=flexible,
  breaklines=true,
  showstringspaces=false
}

\tcbuselibrary{skins, breakable, listings}

\begin{document}

\maketitle

\section*{Learning Strategy}
The goal is to shift from passive learning to active coding. Each task builds on the previous one and introduces key Python fundamentals step-by-step. The learner is expected to code, test, and reflect on their work.

\bigskip

\begin{tcolorbox}[title=Step-by-Step Progression, colback=blue!10, colframe=blue!40, breakable]
\begin{itemize}
    \item \textbf{Mini ATM — Deposit and Withdraw (from scratch)}
    \begin{itemize}
        \item Ask the user what they want to do: Deposit / Withdraw / Quit.
        \item Update the balance according to the action.
        \item Print the current balance after each action.
        \item Loop until user quits.
    \end{itemize}

    \item \textbf{Add Transaction History}
    \begin{itemize}
        \item Store each transaction in a list as a string, e.g. \texttt{"Deposit: +50€"}.
        \item When the user quits, print the list of transactions.
    \end{itemize}

    \item \textbf{Vending Machine}
    \begin{itemize}
        \item Show a menu of items and their prices.
        \item Ask the user to select one and enter their money.
        \item If enough: print a success message and return change.
        \item If not: print an error message.
    \end{itemize}
\end{itemize}
\end{tcolorbox}

\bigskip

\section*{Coding Template for Task 1}
The following scaffold can help the learner get started:

\begin{lstlisting}[language=Python, caption={Basic ATM Structure}]
balance = 0

while True:
    print("What do you want to do?")
    print("1. Deposit")
    print("2. Withdraw")
    print("3. Quit")
    choice = input("> ")

    if choice == "1":
        # TODO: Ask for amount and add to balance
        pass

    elif choice == "2":
        # TODO: Ask for amount, check if enough, subtract
        pass

    elif choice == "3":
        print("Goodbye!")
        break

    else:
        print("Invalid option.")

    # TODO: Print current balance
\end{lstlisting}

\bigskip

\section*{Guiding Questions}
\begin{tcolorbox}[colback=green!10, colframe=green!40, title=Encourage Thinking]
\begin{itemize}
    \item What information does your program need to remember?
    \item What kind of input do you expect from the user?
    \item What conditions should trigger each part of the logic?
    \item How can you organize repeated actions inside a loop?
    \item What should happen if the user makes a mistake?
\end{itemize}
\end{tcolorbox}

\bigskip

\section*{Next Steps}
Once Task 1 and Task 2 are mastered, gradually introduce:
\begin{itemize}
    \item \textbf{Saving data to a file} (with \texttt{open()}, \texttt{write()}, \texttt{readlines()})
    \item \textbf{Datetime stamps} in transaction history
    \item \textbf{Modular functions} like \texttt{deposit()}, \texttt{withdraw()}
    \item (Later) object-oriented structure with a simple \texttt{ATM} class
\end{itemize}

\end{document}